\documentclass[12pt]{article}
\usepackage{graphicx}
\usepackage{fullpage}
\usepackage{hyperref}
\usepackage{fancyhdr}
\usepackage{setspace}
\pagestyle{fancy}
\fancyhead{}
\fancyfoot{}
\fancyfoot[C]{-\thepage-}
\fancyfoot[L]{ID:997900158}
\fancyfoot[RO]{MPM406 - Case Control Paper}
\renewcommand{\footrulewidth}{0.4 pt}
\renewcommand{\headrulewidth}{0 pt}

\hypersetup{colorlinks = true, linkcolor = blue, citecolor = blue}
\title{investigation into personal hygeine and infection with \emph{Cryptosporidium parvum} in a teaching facility in The california central valley}
\author{Student ID: 997900158}
\date{\today}
\begin{document}
	\maketitle
	\begin{abstract}
		This its ht 

	\end{abstract}

\doublespacing
	\section{Introduction} 
	\emph{C. parvum} is a protozoal parasite that can infect the intestinal tract of over causing cryptosprotidiosis in  70 species, \cite{Casemore1997} including humans.
	It is an important human pathogen, with 8,269 cases reported in the USA in 2005, commonly in in patients without normal imune systems, such as the young, old and immunocompromised (e.g. AIDS patients) \cite{Yoder2007}


	Because of the cross infection of \emph{C. parvum}, Cattle are an especially important source of human infection in Calfornia, where there are over 5 million beef and dairy cows. \cite{WesternFarm}
	In cattle infection is also mild and commonnly effects calves between ages of 4 days and 4 weeks \cite{malmo2010}. 
	In beef herds in California, ~4\% of cattle were found to be shedding \emph{C. parvum} in a recent study, with prevlence among calves less than two months old more than 41 times older animals \cite{Atwill1999a}.
	Dairy calves are also at risk of \emph{C. parvum} infection, especially due to managment methods commonly used rearing dairy ehifers. The year round precence of young calves in an intensive managment environment allows constant transmisison and constant endemic infection .\cite{Atwill1998}
	
	
	Transmission between cattle and humans occurs from direct contact with animal faeces that contain the infections \emph{C. parvum} oocysts\cite{malmo2010}.
	The ooccysys are hardy and can survive in the envirnonemnt for some time, depending on environmentla conditions present. 
	During hot dry conditions oocyts survival is reduced, although they may persist for some time if microenvironment (e.g. within moist cow pat) is favourable \cite{Robertson1992}.
	Preventiing tranmission of cryptosporidiosis from infected cattle faeces is based on good sanitation and hygiene practices\cite{malmo2010}, although water contamination can occut and was responsible for a large outbreak in miluakee in 1993 \cite{Kenzie1994}. 
	\emph{C. parvum} oocycysts hae been located downstream from cattle production facilities and contamination of watersheds is a concern from a human helth persepective \cite{Ong1996}. 


	California ranchers and their families are exposed to \emph{C. parvum} oocysts in their work and home environments, with contamination of drinking and recreational water sources been a major source of this exposure.
	Recreational pursuits common in the ranching community include hiking, fishing, swimmin and water skiing, allp potential sources of exposure to contaminated water. 
	Many california ranching households also have recreational pools to escape the summer heat, and \emph{C parvum} oocysts have been shown to be chlorine resistant, especially in poorly sanitised pools \cite{Carpenter1999}.
	Ranchers commonly grow their own food, and this may be a source of infection with \emph{C. parvum} oocysts if contaminated water is used for irrigation purposes.
	No studies we are aware of have investigated the link between contact with contaminated water sources and home food production with cryptosporidiosis. 


	It was hypothesised that ranchign families experience higher levels of \emph{C. parvum} exposure through direct and indirect contact with cattle faeces, and this esposure is responsible for an increased incidence of cryptosporidiosis in these households.

		
	The objective of this study is to identify risk factors assocaited with cryptosporidiosis in ranching families, by comparing the incidence or cryptosporidiosis in ranching hosueholds with different levels of occumaptional, dietry, and household factors.
	The study hypothesis for this objective is some ranching families are at higher risk of cryptosporidiosis due to occupational, dietry and household factors.


	\section{Methods} 
		The study population was ranchers in california 
	The unit of study for this study is the household, and the outcome is incidence of cryptosporidiosis in cases per unit time.


		Ranchers were identified from USDA databases as having active ranching operations (sold cattle in state recorded sales in the previous 12 motnhs)
		This sample (n=1923) was mailed a letter explaining study design and purpose, and a brief return questinoare to expres interest in participting and basic demographic data 
		The high return rate for this initial letter (n= 1704, 88.61 \%) demonstrates the willingless of ranchers to participate, and the effectivnenss of previous extension efforts to riase awareness of this important zoonotic disiase. 
		The initial demographic data gathered included household paramaters as listed in table 1, as well as extimated calving start date (CSD).


		Based on these inital responses, a local nurse practitioner organised a mutually agreeable time to visit ranching properties, at least 2 months before CSD.	% collect cow faeces samples here? depend on time of year?
		For each study area the nurse practitioner was selected from the local bush nursing facility, and where possible preference was given to those that had been in the community for the longest period of time.


		During this visit the NP sought to confirm as many of the details collected in the inital questionare as possible, as well as gather information regarding the property layout ( were there cattle in close proximity to home block), and determined source of drinking water for each property.
		At the conclustion of this visit the nurse practitioner also dropped off faeces collection materials and prepaid return envelopes to failitate sample detection in symptomatic individuals. 		% could freeze amples on site then collect at end of season at same time as interview for case.
		Training was provided in correct sample collection and handling procedures.
		These samples were mailed to the CAHS lab at UC Davis for IFA and DNA isolation,  as described in \cite{Atwill1999}
		

		Cases of intestinal discomfort, fever, and diarrhoea were self reported by ranching families, and confimred with IFA and DNA isolation on stool sample to confirm \emph{Cryptosporidium parvum} was the causitaive agent.
		Study participants names were also searched in local medical databases to detect any cases that went unreported. 15 cases in the exposed group and 18 in the unexposed group were identified in this way. 
		
		No matching was undertaken due to the sparsity of study population with respect to predictor variables. Blinding NP was unneccecary as at the time of interview, the case outcome of each family was unknown.


		Where possible, stocking rates were confirmed by crossreferencing sales records with property registration records
		
		inclusion criteria
			any immuno compromised people were excludedd from the study
			any without complete record.

	\subsection{Statistical Evaluation}
		household level random effect.

	\section{Results}
		During the study period there were XXXX cases of crypto sporidiosis confirmed by fecal IFA, giving an overall incidence density rate of XXXX per 100,000 person years. 
		This high incidence rate contrasts with the general population ( CITATION ) 
		Within the ranching comuunity, there were various risk factors identified for cryptosporidiosis , as listed in Table 2. 


	\section{Discussion} 

		this study improves on previous by including region specific climate data \cite{CIMIS} to predict oocyte survival. 

		reducing calving season length will reduce likelyhood of infection, and is also better for pasture utilisation and overall operation efficency (CITE)	

	\subsection{Strengths and Limitations}
		

		using the same lab for all test ensures internal validity
		Using a nurse practitioner known to the local community rather than a foriegn researcher improves quality of resposnses by facilitating trust between study participants and data collection point. people are more likely to share true information with someone they see at the postoffice weekly, and in turn the nurse practitioner will already have a wealth of information about local families and ranching properties, and can use judgement when deciding accuary of responses
		The physical visit by nurse practitioner greatly imporves data quality. during this visit she was able to inspect and objectively record property layout, water sources, precence of pool on property which imporves the wuality of exposure measures.
		the use of county medical records to confirm cases where medical attention was sought improves the accuracy of outcome measurement, although the majority (n=XXX) of cases were only confirmed through IFA on submitted fecal samples, a demonstration of the average ranchers stoicness and difficulty in accessing medical care in remte rural communities. 

		\flushpage
		
	\section{Figures}

\begin{figure*}[h!]
	\centering
	\includegraphics[scale=0.4]{figure1.jpg}
	\caption{Flow Diagram showing proposed Biological Rationale for study, including exposure, outcome and covariates }
	\label{fig:1}
\end{figure*}

\begin{figure*}[h!]
	\centering
	\includegraphics[scale=0.5]{table1.jpg}
	\caption{Characteristics of study participants and sample size calculations.}
	\label{tab:1}
\end{figure*}
 
\begin{figure*}[h!]
	\centering
	\includegraphics[scale=0.7]{table2.jpg}
	\caption{Odds Ratios (OR) for the association between uveitis and \emph{Bartonella sp.} infection status, age, housing status and geographical location.}
	\label{tab:2}
\end{figure*}

\begin{figure*}[h!]
	\centering
	\includegraphics[scale=0.6]{samplesizecalc.jpg}
	\caption{Sample size function and calculation output from R. Calculations agrees with Epi Info when continuity correction was applied.}
	\label{fig:samplesizecalc}
\end{figure*}

\clearpage
\bibliographystyle{unsrt}
\bibliography{crypto}



\end{document}

%%%%%

% interactoins - 
%% make streamlined as possible. but need at least one confounder, not necc any interactions, can say did not find any (kim paper for technique - none stat sig so not included). potential effect modify - wualitative. 
% if include need to show OR with and without interactions.
% table 1 - cats in study,. throw in other factors that wont include - make them the same. e.g. age w exposure outcome and not on causal path. - no sig p value. 
% fig 1 - exposure b \emph{Bartonella sp.} , outcome ev.

% diagram bartonella causing uveitis, age associated w bartonella and uveitis (double ended arrow.) then show in table.( anythign in table that same do not need to have in diagram.

%%%%%%% - new notes

%  ascertain exposure from cahfs records and casses case w medical records? - access
%% cases w survey?
% temporal bias?

% does study test Ho?
% infection varies w location, season, year?? precipitation as covariate.
