\documentclass[12pt]{article}
\usepackage{graphicx}
\usepackage{hyperref}
\usepackage{fancyhdr}
\usepackage{setspace}
\pagestyle{fancy}
\fancyhead{}
\fancyfoot{}
\fancyfoot[C]{-\thepage-}
\fancyfoot[L]{ID:997900158}
\fancyfoot[RO]{MPM406 - Case Control Paper}
\renewcommand{\footrulewidth}{0.4 pt}
\renewcommand{\headrulewidth}{0 pt}

\hypersetup{colorlinks = true, linkcolor = blue, citecolor = blue}
\title{Investigation into the effect of growing food onsite on the incidence of cryptosporidiosis in California Ranching households}
\author{Student ID: 997900158}
\date{\today}
\begin{document}
	\maketitle
	\begin{abstract}
		This cohort study's objective was to investigate the relationship between California ranchers growing food onsite and incidence of cryptosporidiosis in their household.
		Ranchers from a USDA were sent an initial questionnaire, followed up by a visit from a nurse practitioner to collect information on various covariates.
		After adjusting for source of drinking water, households that grew more than 50\% their food  on site were 5.1 times more likely to experience a cryptosporidiosis event each year than those who did not grow more than 50\% of food onsite.

	\end{abstract}

\onehalfspace
	\section{Introduction} 
	\emph{C. parvum} is a protozoal parasite that can infect the intestinal tract of over causing cryptosprotidiosis in  70 species, \cite{Casemore1997} including humans.
	It is an important human pathogen, with 8,269 cases reported in the USA in 2005, commonly in in patients without normal immune systems, such as the young, old and immunocompromised (e.g. AIDS patients) \cite{Yoder2007}


	Because of the cross infection of \emph{C. parvum}, Cattle are an especially important source of human infection in California, where there are over 5 million beef and dairy cows. \cite{WesternFarm}
	In cattle infection is also mild and commonly effects calves between ages of 4 days and 4 weeks \cite{malmo2010}. 
	In beef herds in California, ~4\% of cattle were found to be shedding \emph{C. parvum} in a recent study, with prevalence among calves less than two months old more than 41 times older animals \cite{Atwill1999a}.
	Dairy calves are also at risk of \emph{C. parvum} infection, especially due to management methods commonly used rearing dairy heifers. The year round presence of young calves in an intensive management environment allows constant transmission and constant endemic infection .\cite{Atwill1998}
	
	
	Transmission between cattle and humans occurs from direct contact with animal faeces that contain the infections \emph{C. parvum} oocysts\cite{malmo2010}.
	The oocysts are hardy and can survive in the environment for some time, depending on environmental conditions present. 
	During hot dry conditions oocyst survival is reduced, although they may persist for some time if microenvironment (e.g. within moist cow pat) is favourable \cite{Robertson1992}.
	Preventing transmission of cryptosporidiosis from infected cattle faeces is based on good sanitation and hygiene practices\cite{malmo2010}, although water contamination can occur and was responsible for a large outbreak in Milwaukee in 1993 \cite{Kenzie1994}. 
	\emph{C. parvum} oocysts have been located downstream from cattle production facilities and contamination of watersheds is a concern from a human health perspective \cite{Ong1996}. 


	California ranchers and their families are exposed to \emph{C. parvum} oocysts in their work and home environments, with contamination of drinking and recreational water sources been a major source of this exposure.
	Recreational pursuits common in the ranching community include hiking, fishing, swimming and water skiing, all potential sources of exposure to contaminated water. 
	Many California ranching households also have recreational pools to escape the summer heat, and \emph{C parvum} oocysts have been shown to be chlorine resistant, especially in poorly sanitised pools \cite{Carpenter1999}.
	Ranchers commonly grow their own food, and this may be a source of infection with \emph{C. parvum} oocysts if contaminated water is used for irrigation purposes.
	No studies we are aware of have investigated the link between contact with contaminated water sources and home food production with cryptosporidiosis. 


	It was hypothesised that ranching families experience higher levels of \emph{C. parvum} exposure through direct and indirect contact with cattle faeces, and this exposure is responsible for an increased incidence of cryptosporidiosis in these households.

		
	The objective of this study is to identify risk factors associated with cryptosporidiosis in ranching families, by comparing the incidence or cryptosporidiosis in ranching households with different levels of occupational, dietary, and household factors.
	The study hypothesis for this objective is some ranching families are at higher risk of cryptosporidiosis due to occupational, dietary and household factors.


	\section{Methods} 
		The study population was ranchers in California 
	The unit of study for this study is the household, and the outcome is incidence of cryptosporidiosis in cases per unit time.


		Ranchers were identified from USDA databases as having active ranching operations (sold cattle in state recorded sales in the previous 12 months)
		This sample (n=1923) was mailed a letter explaining study design and purpose, and a brief return questionnaire to express interest in participating and basic demographic data 
		The high return rate for this initial letter (n= 1704, 88.61 \%) demonstrates the willingness of ranchers to participate, and the effectiveness of previous extension efforts to raise awareness of this important zoonotic disease. 
		The initial demographic data gathered included household parameters as listed in table 1, as well as estimated calving start date (CSD).


		Based on these initial responses, a local nurse practitioner organised a mutually agreeable time to visit ranching properties, at least 2 months before CSD.	% collect cow faeces samples here? depend on time of year?
		For each study area the nurse practitioner was selected from the local bush nursing facility, and where possible preference was given to those that had been in the community for the longest period of time.


		During this visit the NP sought to confirm as many of the details collected in the initial questionnaire as possible, as well as gather information regarding the property layout ( were there cattle in close proximity to home block), and determined source of drinking water for each property.
		At the conclusion of this visit the nurse practitioner also dropped off faeces collection materials and prepaid return envelopes to facilitate sample detection in symptomatic individuals. 		% could freeze amples on site then collect at end of season at same time as interview for case.
		Training was provided in correct sample collection and handling procedures.
		These samples were mailed to the CAHS lab at UC Davis for IFA and DNA isolation,  as described in \cite{Atwill1999}
		At the end of the study period, participants were mailed a final questionnaire regarding GI upsets.
		

		Cases of intestinal discomfort, fever, and diarrhoea were self reported by ranching families, and confirmed with IFA and DNA isolation on stool sample to confirm \emph{Cryptosporidium parvum} was the causative agent.
		Study participants names were also searched in local medical databases to detect any cases that went unreported. 15 cases in the exposed group and 18 in the unexposed group were identified in this way. 
		
		No matching was undertaken due to the sparsity of study population with respect to predictor variables. Blinding NP was unnecessary as at the time of interview, the case outcome of each family was unknown.
		Each individuals medical history was examined and any with incidence of severe diarrhoea or other gastrointestinal disease in the last 6 moths was excluded. Any Individual with a history of compromised immune system was removed from the study.
		As the duration of shedding \emph{C. parvum} is typically below one month following infection, it was decided an examination of medical history for past gastrointestinal problems was sufficient for screening prevalent cases of disease, as individual fecal samples at enrollment would prove too costly and would scare many potential study participants away.
		


		Sample size requirements were computed using Epi info with alpha of 0.05, desired power of 0.8, biologically significant RR of 2, returning a sample size per group of 686 per group, with 10\% added for attrition, to return total sample size of 1510.
		Output can be seen in \ref{samplesizecalc}

		Where possible, stocking rates were confirmed by cross-referencing sales records with property registration records
		

		Data quality was maintained throughout the study by using trained professionals for data collection, querying county medical records to check for missed cases, and using barcoded sample collection containers unique to each household to eliminate mistakes in the lab.
		IFA tests were calibrated against a gold standard before and after running each batch to ensure consistency.

	\subsection{Statistical Evaluation}
		Cox proportional hazard models were used with time to event as response and various covariates discussed above as predictors, with a household level random effect included.
		The baseline hazard was modified to follow a weibull distribution to reflect the fact that risk for cryptosporidiosis varies with time over the season, with an increased risk occurring with the presence of young calves following calving start date. 
		This model yield results as shown in \ref{table2}
		Due to some loss of study households in both case and control groups, Incidence density rates were calculated for all strata of the covariates with findings indicated in \ref{table2}.



	\section{Results}
		During the study period there were 1042 cases of cryptosporidiosis confirmed by fecal IFA among people who grew more than 50\% of their own food, giving an overall incidence density rate of 62 cases per 100 person years.
		Within the ranching community, there were various risk factors identified for cryptosporidiosis , as listed in \ref{table2}. 
		Number of aged people, presence of a pool, slaughtering own meat, eating wild game, size of property, and number of animals were all found to not differ significantly between the households that experienced a cryptosporidiosis event and those that did not. 


		A potential confounder investigated in this study was age.
		Age of case was recorded and included investigated as it was known children are more likely to become infected with cryptosporidiosis, and are also more likely to be exposed by swimming in contaminated waterways, and practicing substandard personal hygiene.
		It was found that IDR for households with young children differed from crude unadjusted IDR, hence the IDR was adjusted for age in the final model.


		Interactions that were investigated included source of drinking water and more than half of food grown on site.
		households that used surface water for drinking and grew more than 50\% of food on site were more likely to undergo an episode of cryptosporidiosis, and hence the RR found in final model were adjusted for this interaction.


		Sources of possible bias in this study was selection bias and follow up bias.
		It is possible that there was differential loss to follow up during this study, with ranching families that did not experience a any gastrointestinal event during study period not bothering to report at end of study period.
		It is also possible that not all cases of GI upset due to cryptosporidiosis were recorded and confirmed. 
		Selection bias may have occurred as families with a history of GI upset may be more interested in study finding ans hence be more likely to participate. as there is some evidence of immunity following initial episode, this may reduce incidence in exposure group, reducing calculated RR.


	\section{Discussion} 


	\subsection{Strengths and Limitations}
		Two strengths of the study are the data quality and accuracy of exposure information.

		Using a nurse practitioner known to the local community rather than a foreign researcher improves quality of responses by facilitating trust between study participants and data collection point. 
		People are more likely to share true information with someone they see at the postoffice weekly, and in turn the nurse practitioner will already have a wealth of information about local families and ranching properties, and can use judgement when deciding accuracy of responses
		The physical visit by nurse practitioner greatly improves the accuracy of exposure information. 
		During this visit she was able to inspect and objectively record property layout, water sources, presence of pool on property which improves the quality of exposure measures.
		The use of county medical records to confirm cases where medical attention was sought improves the accuracy of outcome measurement, although the majority of cases were only confirmed through IFA on submitted fecal samples, a demonstration of the average ranchers stoicis and difficulty in accessing medical care in remote rural communities. 

		Two weaknesses of this study is the loss of some study participants to follow up, and potential outcome recording issues.
		There were 126 households lost to follow through this study. They were either unable to be contacted, or refused to participate at the end of the study period. There is the potential that this was a differential loss to follow due to the study design. 
		Households that did not experience a cryptosporidiosis event may have thought it unnecessary to record or submit samples, which would have resulted in an over reporting of the relative risk.
		Ranching households are stronger than the average household, and will be more likely to deal with a mild GI upset than their city counterparts. 
		\clearpage
		
	\section{Figures}

\begin{figure*}[h!]
	\centering
	\includegraphics[scale=0.4]{figure1.jpg}
	\caption{Flow Diagram showing proposed Biological Rationale for study, including exposure, outcome and covariates }
	\label{fig:1}
\end{figure*}

\begin{figure*}[h!]
	\centering
	\includegraphics[scale=0.5]{table1.jpg}
	\caption{Characteristics of study participants and sample size calculations.}
	\label{tab:1}
\end{figure*}
 
\begin{figure*}[h!]
	\centering
	\includegraphics[scale=0.7]{table2.jpg}
	\caption{Odds Ratios (OR) for the association between uveitis and \emph{Bartonella sp.} infection status, age, housing status and geographical location.}
	\label{tab:2}
\end{figure*}

\begin{figure*}[h!]
	\centering
	\includegraphics[scale=0.6]{sample.jpg}
	\caption{Sample size function and calculation output from R. Calculations agrees with Epi Info when continuity correction was applied.}
	\label{fig:samplesizecalc}
\end{figure*}

\clearpage
\bibliographystyle{unsrt}
\bibliography{crypto}



\end{document}

%%%%%

% interactoins - 
%% make streamlined as possible. but need at least one confounder, not necc any interactions, can say did not find any (kim paper for technique - none stat sig so not included). potential effect modify - wualitative. 
% if include need to show OR with and without interactions.
% table 1 - cats in study,. throw in other factors that wont include - make them the same. e.g. age w exposure outcome and not on causal path. - no sig p value. 
% fig 1 - exposure b \emph{Bartonella sp.} , outcome ev.

% diagram bartonella causing uveitis, age associated w bartonella and uveitis (double ended arrow.) then show in table.( anythign in table that same do not need to have in diagram.

%%%%%%% - new notes

%  ascertain exposure from cahfs records and casses case w medical records? - access
%% cases w survey?
% temporal bias?

% does study test Ho?
% infection varies w location, season, year?? precipitation as covariate.
