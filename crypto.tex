\documentclass[12pt]{article}
\usepackage{graphicx}
\usepackage{hyperref}
\usepackage{fancyhdr}
\pagestyle{fancy}
\fancyhead{}
\fancyfoot{}
\fancyfoot[C]{-\thepage-}
\fancyfoot[L]{ID:997900158}
\fancyfoot[RO]{MPM406 - Case Control Paper}
\renewcommand{\footrulewidth}{0.4 pt}
\renewcommand{\headrulewidth}{0 pt}

\hypersetup{colorlinks = true, linkcolor = blue, citecolor = blue}
\title{investigation into personal hygeine and infection with \emph{Cryptosporidium parvum} in a teaching facility in The california central valley}
\author{Student ID: 997900158}
\date{\today}
\begin{document}
	\maketitle
	\begin{abstract}


	\end{abstract}


	\section{Introduction} 
	age of calf is strongly associated with the probability of shedding \emph{Cryptosporidium parvum} \cite{Atwill1999a}.
	precence of young calves , e.g. in dairy where young calves are present year round allows constant transmisison and constant endemic infection .\cite{Atwill1998}
	environmental effects oocyte survival and 
	during hot dry conditions oocyts survive for less, although this may be altered by physical location - e.g. within moist cow pat. \cite{Robertson1992}
		
	Objectives: identify risk factors assocaited with \emph{Cryptosporidium parvum} infection and
	Ho: domestic, lifestyle, and family facors exist that predispose some families from having a higher 
 		
	\section{Methods} 
		Questionares were mailed to ranchers in california sourced off USDA databases of active ranching operations having sold cattle in the previous 12 motnhs.
		questions included normal saclving start date.
		based on these inital responses, a nurse practitioner organised a mutually agreeable time to visit ranching properties, at least 2 months before CSD.	% collect cow faeces samples here? depend on time of year?
	during this inital visit the nurse practitioner dropped off faeces collection materials and prepaid return envelopes to failitate sample detection in symptomatic individuals.		% could freeze amples on site then collect at end of season at same time as interview for case.
		theses smaples were mailed to the CAHS lab at UC Davis for IFA and DNA isolation,  as described in \cite{Atwill1999}

		These questiounares included items listed in table 1
		also included recreationsl pursuits such as swimming, fishing, skiing on local water bodies, and any hiking activities 
		precence of a pool on premesis as some \emph{Cryptosporidium parvum} strains have been shown to be chlorine resistant, especially in poorly sanitised pools \cite{Carpenter1999}.
		source of drinking water (well, recovered rain water, other)

		number of children under age..
		how many hours spouse and children spent assisting primary rancher 
		
		stocking rate was calculated for all animals and for animals under 4 months of age.	% but as beef need momma, no solo calves. can count cow-calf pairs as in his javma study?\cite{Atwill1999}

		case definitions:
		cases were self reported, and cross referenced with local medical databases.
		
		surnames and addressed were also queried from the database to identify any missed cases, and thses mised cases followed up with a phone call.
		



		Stocking rates were confirmed by crossreferencing sales records with property registration records
		
		inclusion criteria
			any immuno compromised people were excludedd from the study
			any without complete record.

	\subsection{Statistical Evaluation}
		hed level random effect.

	\section{Results}
	

	\section{Discussion}


		using the same lab for all test ensures internal validity
		the same nurse practitioner completed all qustionares ensuring a consistent delivery and level of detail across all study participants
		

		reducing calving season length will reduce likelyhood of infection, and is also better for pasture utilisation and overall operation efficency (CITE)	

	\subsection{Strengths and Limitations}

\begin{figure*}[h!]
	\centering
	\includegraphics[scale=0.4]{figure1.jpg}
	\caption{Flow Diagram showing proposed Biological Rationale for study, including exposure, outcome and covariates }
	\label{fig:1}
\end{figure*}

\begin{figure*}[h!]
	\centering
	\includegraphics[scale=0.5]{table1.jpg}
	\caption{Characteristics of study participants and sample size calculations.}
	\label{tab:1}
\end{figure*}
 
\begin{figure*}[h!]
	\centering
	\includegraphics[scale=0.7]{table2.jpg}
	\caption{Odds Ratios (OR) for the association between uveitis and \emph{Bartonella sp.} infection status, age, housing status and geographical location.}
	\label{tab:2}
\end{figure*}

\begin{figure*}[h!]
	\centering
	\includegraphics[scale=0.6]{samplesizecalc.jpg}
	\caption{Sample size function and calculation output from R. Calculations agrees with Epi Info when continuity correction was applied.}
	\label{fig:samplesizecalc}
\end{figure*}

\clearpage
\bibliographystyle{unsrt}
\bibliography{bart.bib}



\end{document}

%%%%%

% interactoins - 
%% make streamlined as possible. but need at least one confounder, not necc any interactions, can say did not find any (kim paper for technique - none stat sig so not included). potential effect modify - wualitative. 
% if include need to show OR with and without interactions.
% table 1 - cats in study,. throw in other factors that wont include - make them the same. e.g. age w exposure outcome and not on causal path. - no sig p value. 
% fig 1 - exposure b \emph{Bartonella sp.} , outcome ev.

% diagram bartonella causing uveitis, age associated w bartonella and uveitis (double ended arrow.) then show in table.( anythign in table that same do not need to have in diagram.

%%%%%%% - new notes

%  ascertain exposure from cahfs records and casses case w medical records? - access
%% cases w survey?
% temporal bias?

% does study test Ho?
% infection varies w location, season, year?? precipitation as covariate.
